%% LyX 2.1.4 created this file.  For more info, see http://www.lyx.org/.
%% Do not edit unless you really know what you are doing.
\documentclass[english]{article}
\usepackage[T1]{fontenc}
\usepackage[latin9]{inputenc}
\usepackage[letterpaper]{geometry}
\geometry{verbose,tmargin=2cm,bmargin=2cm,lmargin=4.5cm,rmargin=4.5cm}
\usepackage{amsbsy}
\usepackage{graphicx}
\usepackage{esint}

\makeatletter

%%%%%%%%%%%%%%%%%%%%%%%%%%%%%% LyX specific LaTeX commands.
%% Because html converters don't know tabularnewline
\providecommand{\tabularnewline}{\\}

%%%%%%%%%%%%%%%%%%%%%%%%%%%%%% Textclass specific LaTeX commands.
\newcommand{\lyxaddress}[1]{
\par {\raggedright #1
\vspace{1.4em}
\noindent\par}
}

\makeatother

\usepackage{babel}
\begin{document}

\title{Acoustic modal expansion of open cavity using coupled mode theory}


\author{Yuhui Tong, Yiwei Kou, Jie Pan}

\maketitle

\lyxaddress{Schoold of Mechanical and Chemical Engineering, The University of
Western Australia, Crawley, WA 6009, Australia}


\section{Introduction }

Open cavity widely ... problem

Intensive study eigenvalue problem, the modal expansion has long been
absence. 

Several attempts done before, but none solve the problem nor did provide
any information to verify the ...

A recent progress made in acoustic resonators for using coupled mode
theory, to . It is also proved that with frequency-dependent eigenmodes. 

In the present paper, it is extended to open cavity problem. The baffled
being treated ... 


\section{Theory}


\subsection{Phase I, problem formulation}

As depicted in Fig. \ref{fig:baffled-cavity}, the rectangular cavity
with an opening located at an infinite baffle is investigated in the
present paper. For the sake of simplicity, all boundaries are rigid. 

\begin{figure}
\begin{centering}
\includegraphics[scale=0.5]{\string"E:/yuhui/OneDrive/YUHUI_UWA/2D_QNM/coupled mode theory/opencavity/opencavity_theory/opencavity_illustration\string".png}
\par\end{centering}

\caption{\label{fig:baffled-cavity}A schematic diaggram of an baffled open
cavity.}
\end{figure}


The sound pressure (omitting the time dependence $e^{j\omega t}$)
excited by a monopole source inside the cavity can be obtained by
solving the inhomogeneous Helmholtz equation

\begin{equation}
\nabla^{2}p+k^{2}p=-q_{s}\delta(\boldsymbol{x}-\boldsymbol{x}_{s}),\label{eq:Helmholtz eqaution}
\end{equation}
where $k=\omega/c_{0}$ is the wavenumber, $q_{s}$ is the source
strength, $\boldsymbol{x}_{s}$ is the position of the monopole source,
which is supposed to be within the cavity. The solution of Eq. (\ref{eq:Helmholtz eqaution})
can be written respectively for cavity area $\Omega_{a}$, and upper
half space $\Omega_{b}$. For $\Omega_{a}$,

\begin{equation}
p_{a}(\boldsymbol{x})=\sum_{\mu,\nu,\xi}a_{\mu,\nu,\xi}\phi_{\mu,\nu,\xi}(\boldsymbol{x}),\label{eq:pressure_a}
\end{equation}
where $\phi_{\mu',\nu',\xi'}(\boldsymbol{x})=\psi_{\mu'}(x)\psi_{\nu'}(y)\psi_{\xi'}(z)$
is the eigenmode for \emph{enclosed} cavity and $\psi_{m}(x)=\sqrt{\frac{2-\delta_{0,m}}{L_{x}}}\cos(\frac{m\pi}{L_{x}}x)$.
For $\Omega_{b}$, 
\begin{equation}
p_{b}(\boldsymbol{x})=jk\rho c\iint_{S_{0}}G_{b}(\boldsymbol{x},\boldsymbol{x}')v_{\perp}(\boldsymbol{x}')dS',\label{eq:pressure_b}
\end{equation}
where $G_{b}(\boldsymbol{x},\boldsymbol{x}')=\frac{1}{2\pi}\frac{e^{-jk|\boldsymbol{x}-\boldsymbol{x}'|}}{|\boldsymbol{x}-\boldsymbol{x}'|}$
is the Green's function for the upper half space, $v_{\perp}(\boldsymbol{x}')$
is the normal velocity distribution at the opening $S_{0}$ (the intersection
between $\Omega_{a}$ and $\Omega_{b}$), the integral is invaluated
over $S_{0}$.

In $\Omega_{a}$, $p_{a}$ satisfies,

\begin{equation}
\nabla^{2}p_{a}(\boldsymbol{x})+k^{2}p_{a}(\boldsymbol{x})=-q_{s}\delta(\boldsymbol{x}-\boldsymbol{x}_{s}),\label{eq:pa}
\end{equation}
together with corresponding boundary conditions, while $\phi_{\mu,\nu,\xi}$
satisfies,

\begin{equation}
\nabla^{2}\phi_{\mu,\nu,\xi}+k_{\mu,\nu,\xi}^{2}\phi_{\mu,\nu,\xi}=0,\label{eq:phi}
\end{equation}
\[
k_{\mu,\nu,\xi}^{2}=(\mu\pi/L_{x})^{2}+(\nu\pi/L_{y})^{2}+(\xi\pi/L_{z})^{2},
\]
and rigid boundary condition for all six walls of the rectangular
cavity, including $S_{0}$. Multiplying Eq. (\ref{eq:pa}) and Eq.
(\ref{eq:phi}) by $\phi_{\mu,\nu,\xi}$ and $p_{a}$ respectively
and taking the difference of the resulting equations yields,

\begin{equation}
\left(p_{a}\nabla^{2}\phi_{\mu,\nu,\xi}-\phi_{\mu,\nu,\xi}\nabla^{2}p_{a}\right)+(k_{\mu,\nu,\xi}^{2}-k^{2})p_{a}\phi_{\mu,\nu,\xi}=q_{s}\phi_{\mu,\nu,\xi}\delta(\boldsymbol{x}-\boldsymbol{x}_{s}).
\end{equation}


Integrating over $\Omega_{a}$ and applying Green's theorem gives
\begin{equation}
jk\rho c\iint_{S_{0}}\phi_{\mu,\nu,\xi}v_{\perp}dS_{0}+(k_{\mu,\nu,\xi}^{2}-k^{2})a_{\mu,\nu,\xi}=q_{0}\phi_{\mu,\nu,\xi}(\boldsymbol{x}_{s}).
\end{equation}


The above equation can be further simplied into

\begin{equation}
\sum_{m,n}jk\psi_{\xi}(0)\delta_{\mu,m}\delta_{\nu,n}\rho_{0}c_{0}V_{m,n}+(k_{\mu,\nu,\xi}^{2}-k^{2})a_{\mu,\nu,\xi}=q_{0}\phi_{\mu,\nu,\xi}(\boldsymbol{x}_{s}),\label{eq:constrain1}
\end{equation}
by noting the expansion of $v_{\perp}(x,y)$

\begin{equation}
v_{\perp}(x,y)=\sum_{m,n}V_{m,n}\psi_{m}(x)\psi_{n}(y).
\end{equation}


Another constraint is the continuity condition for sound pressure
at the interface, i.e., $\left.p_{a}\right|_{S_{0}}=\left.p_{b}\right|_{S_{0}}$
such that

\begin{equation}
\sum_{\mu',\nu',\xi'}a_{\mu',\nu',\xi'}\psi_{\mu'}(x)\psi_{\nu'}(y)\psi_{\xi'}(0)=jk\rho c\iint_{S_{0}}\frac{e^{-jk\sqrt{(x-x')^{2}+(x-y')^{2}}}}{2\pi\sqrt{(x-x')^{2}+(x-y')^{2}}}\sum_{m,n}V_{m,n}\psi_{m}(x')\psi_{n}(y')dS_{0}
\end{equation}


Multiplying $\psi_{\mu}(x)\psi_{\nu}(y)$ and integrate over the interface
leads to,

\begin{equation}
\sum_{\mu',\nu',\xi'}\delta_{\mu,\mu'}\delta_{\nu,\nu'}\psi_{\xi'}(0)a_{\mu',\nu',\xi'}=\rho_{0}c_{0}\sum_{m,n}Z_{\mu,\nu,m,n}V_{m,n},\label{eq:constraint2}
\end{equation}
where $Z_{\mu,\nu,m,n}$ is the radiation impedance of an baffled
rectangular plate of $L_{x}\times L_{y}$ 
\begin{equation}
Z_{\mu,\nu,m,n}=jk\iint_{S_{0}}\iint_{S_{0}}\psi_{\mu}(x)\psi_{\nu}(y)\frac{e^{-jk\sqrt{(x-x')^{2}+(x-y')^{2}}}}{2\pi\sqrt{(x-x')^{2}+(x-y')^{2}}}\psi_{m}(x')\psi_{n}(y')dS'dS
\end{equation}


Using Eqs. (\ref{eq:constrain1}) and (\ref{eq:constraint2}), vectors
$\boldsymbol{a}=\left[\begin{array}{ccc}
\cdots & a_{\mu,\nu,\xi} & \cdots\end{array}\right]^{T}$ and $\boldsymbol{V}=\left[\begin{array}{ccc}
\cdots & V_{m,n} & \cdots\end{array}\right]^{T}$can be determined by solving

\begin{equation}
\boldsymbol{H}\boldsymbol{V}+(\boldsymbol{K}-k^{2}\boldsymbol{I})\boldsymbol{a}=\boldsymbol{S}\label{eq:inner_equation}
\end{equation}


\begin{equation}
\boldsymbol{M}\boldsymbol{a}=\boldsymbol{Z}\boldsymbol{V}\label{eq:Inner_Outer_connection}
\end{equation}
where corresponding matrices are defined as follows: $\boldsymbol{H}_{(\mu,\nu,\xi),(m,n)}=jk\delta_{\mu,m}\delta_{\nu,n}\psi_{\xi}(0)$,
$\boldsymbol{K}_{(\mu,\nu,\xi),(\mu',\nu',\xi')}=k_{\mu,\nu,\xi}^{2}\delta_{\mu,\mu'}\delta_{\nu,\nu'}\delta_{\xi,\xi'},$
$\boldsymbol{S}=q_{s}\left[\begin{array}{ccc}
\cdots & \phi_{\mu,\nu,\xi}(\boldsymbol{x}_{s}) & \cdots\end{array}\right]^{T}$, $\boldsymbol{M}_{(m,n),(\mu,\nu,\xi)}=\delta_{\mu,\mu'}\delta_{\nu,\nu'}\psi_{\xi'}(0)$,
$\boldsymbol{Z}_{(\mu,\nu),(m,n)}=Z_{\mu,\nu,m,n}$. Eqs. (\ref{eq:inner_equation})
and (\ref{eq:Inner_Outer_connection}) can be further reduced as 

\begin{equation}
(\boldsymbol{D-k^{2}})\boldsymbol{a}=\boldsymbol{S},\label{eq:sp_governing_eq}
\end{equation}
where $\boldsymbol{D}=\boldsymbol{K}-\boldsymbol{H}\boldsymbol{Z}^{-1}\boldsymbol{M}$
is known as effective Hamiltonian (reduced differential operator)
on quantum physics.


\subsection{Phase II Bi-orthogonal basis and modal expansion.}

Eq. (\ref{eq:sp_governing_eq}) gives rise to the following eigenvalue
problem

\begin{equation}
\boldsymbol{D}\boldsymbol{g}_{\mu,\nu,\xi}=K_{\mu,\nu,\xi}^{2}\boldsymbol{g}_{\mu,\nu,\xi},\label{eq:EVP}
\end{equation}
where $K_{\mu,\nu,\xi}^{2}$ is the eigenvalue and the eigenvector
$\boldsymbol{g}_{\mu,\nu,\xi}$ satisfies the bi-orthogonal relation

\begin{equation}
\boldsymbol{g}_{\mu',\nu',\xi'}^{T}\boldsymbol{g}_{\mu,\nu,\xi}=\delta_{\mu',\mu}\delta_{\nu,\nu'}\delta_{\xi,\xi'}\boldsymbol{g}_{\mu,\nu,\xi}^{T}\boldsymbol{g}_{\mu,\nu,\xi}.\label{eq:bi-orthogonal_relation}
\end{equation}
An alternative expression of Eq. (\ref{eq:bi-orthogonal_relation})
is written as,

\begin{equation}
\iiint_{V_{0}}\Phi_{\mu',\nu',\xi'}(\boldsymbol{x})\Phi_{\mu,\nu,\xi}(\boldsymbol{x})dV=\delta_{\mu',\mu}\delta_{\nu,\nu'}\delta_{\xi,\xi'}\iiint_{V_{0}}\Phi_{\mu,\nu,\xi}^{2}(\boldsymbol{x})dV,
\end{equation}
where $\Phi_{\mu,\nu,\xi}(\boldsymbol{x})$ is the modal function
corresponding to $\boldsymbol{g}_{\mu,\nu,\xi}$ such that 

\begin{equation}
\Phi_{\mu,\nu,\xi}(\boldsymbol{x})=\sum_{\mu',\nu',\xi'}\left(\boldsymbol{g}_{\mu,\nu,\xi}\right)_{\mu',\nu',\xi'}\phi_{\mu',\nu',\xi'}(\boldsymbol{x}).
\end{equation}


Expanding $\boldsymbol{a}$ into $\left\{ \boldsymbol{g}_{\mu,\nu,\xi}\right\} $

\begin{equation}
\boldsymbol{a}=\sum_{\mu',\nu',\xi'}c_{\mu',\nu',\xi'}\boldsymbol{g}_{\mu',\nu',\xi'}\label{eq:eigen_modal_expansion}
\end{equation}
and making the substitution into Eq. (\ref{eq:sp_governing_eq}) yields 

\begin{equation}
c_{\mu,\nu,\xi}=\frac{\boldsymbol{g}_{\mu,\nu,\xi}^{T}\boldsymbol{S}}{(K_{\mu,\nu,\xi}^{2}-k^{2})\boldsymbol{g}_{\mu,\nu,\xi}^{T}\boldsymbol{g}_{\mu,\nu,\xi}}.
\end{equation}
Combining Eq. (\ref{eq:eigen_modal_expansion}) together with Eqs.
(\ref{eq:pressure_a} ,\ref{eq:pressure_b},\ref{eq:Inner_Outer_connection})
leads to the modal expansion of sound pressure of the open cavity,

\[
p(\boldsymbol{x})=\left(\begin{array}{c}
p_{a}(\boldsymbol{x})\\
p_{b}(\boldsymbol{x})
\end{array}\right)=\sum_{\mu,\nu,\xi}c_{\mu,\nu,\xi}\left(\begin{array}{c}
\Phi_{\mu,\nu,\xi}(\boldsymbol{x})\\
\Psi_{\mu,\nu,\xi}(\boldsymbol{x})
\end{array}\right),
\]
where $\Psi_{\mu,\nu,\xi}(\boldsymbol{x})$ is given by

\[
\Psi_{\mu,\nu,\xi}(\boldsymbol{x})=\boldsymbol{\varphi}^{T}\boldsymbol{Z}^{-1}\boldsymbol{M}\boldsymbol{g}_{\mu,\nu,\xi}.
\]
Noted $\boldsymbol{\varphi}=\left[\begin{array}{ccc}
\cdots & \varphi_{m,n}(\boldsymbol{x}) & \cdots\end{array}\right]$, $\varphi_{m,n}(\boldsymbol{x})=jk\rho c\iint_{S_{0}}G_{b}(\boldsymbol{x},\boldsymbol{x}')\psi_{m}(x')\psi_{n}(y')dS'$
is the sound pressure induced by velocity distribution $\psi_{m}(x')\psi_{n}(y')$
on $S_{0}$.


\section{Numerical validation}

The theoretical results obtained in Sec. 2 is check numerically here.
The cavity in Fig. \ref{fig:baffled-cavity} has the dimensions of
0.432m long ($L_{x}$), 0.67m wide ($L_{y}$) and 0.598m high ($L_{z}$),
which was considered in literature (Wang). The source is located at
$(0.1,\ 0.1,\ -L_{z}+0.1)$ m with . The field points inside and outside
the cavity are randomly chosen at $(0.2,\ 0.3,\ -L_{z}+0.4)$ m and
$(1.3,\ 1.4,\ -L_{z}+1.5)$ m. The analytical method proposed in Sec.
2 is obtained with MATLAB codes, when 140 \emph{closed} cavity modes
are used for computation of eigensolutions to Eq. (\ref{eq:EVP}). 

140 eigenmodes are obtained by analytical method for open rectangular
cavity. Table. \ref{tab:eigen_frequencies} lists the first 15 eigen
solutions when noise frequency $f=500$ Hz $(k=9.24)$, in constrast
with the counterparts of closed cavity. It's clear that the eigenfrequencies
of acoustic modes become complex in which the imaginary part corresponds
to radiation loss. Table. \ref{tab:Modulus of modes} plots slices
of $|\Psi_{\mu,\nu,\xi}|$, the modulus of modal function of $(\mu,\nu,\xi)$
eigenmode. The nodal lines are distinguishable for these low frequency
eigenmodes, which justifies the inheritance of the closed cavity modes'
indexes $\mu,\nu,\xi$ to classify the open cavity modes. The bi-orthogonality
of the eigensolutions is checked in Tab. \ref{tab:check_biorthogonality}.
Figure. \ref{fig:Modal_coefficients} presents the amplitude each
eigenmodes upon the monopole source with strength $q_{s}=jk\rho_{0}c_{0}q_{0}$,
and $q_{0}=10^{-4}m^{3}/s$ where one can see that $|c_{\mu,\nu,\xi}|$
decays rapidly as the order of modes grows. $|c_{\mu,\nu,\xi}|$ takes
the maximum value at (1,1,0) mode, of which the eigenfrequency takes
$484.95+1.83j$ Hz. It is then checked the number of eigenmodes needed
for calculation. Figure. \ref{fig:convergence I} indicates less than
15 eigenmodes is necessary for sound pressure to converge at the probe
points inside the cavity, while less than 20 eigenmodes for probe
points outside the cavity. This result is quite reasonable considering
the resonant eigenmode is the $6^{th}$ one. 

The performance of the proposed method is then verified by calculating
the sound pressure at field points inside and outside cavity for multiple
frequencies below 500 Hz, where 20 eigenmodes will be used for calculation.
The The reference result is obtain via finite element software COMSOL,
where perfectly matched layers (PMLs) are used to model the semi-infinite
space above the baffle. Noted only frequencie above 30 Hz are treated
in COMSOL, as at very low frequency, the PMLs needed for calculation
become so thick to prevent reflected wave. The source strength is
taken as $q_{s}=4\pi\times10^{-4}kg/s^{2}$. Figure. \ref{fig:Comparison_FEA_Analytical}
plots the comparison between results obtained by both methods, where
excellent agreement suggests the 

\begin{table}
\begin{centering}
\begin{tabular}{|c|c|c|c|c|}
\hline 
\hline$\mu$ & $\nu$ & $\xi$  & $f_{\mu,\nu,\xi}$ (Hz) & $F_{\mu,\nu,\xi}$ (Hz)\tabularnewline
\hline 
\hline 
0 & 0 & 0 & 0 & 133.26+23.55j\tabularnewline
\hline 
0 & 1 & 0 & 253.73 & 283.08+8.64j\tabularnewline
\hline 
0 & 0 & 1 & 284.28 & 382.88+71.04j\tabularnewline
\hline 
0 & 1 & 1 & 381.04 & 449.37+40.02j\tabularnewline
\hline 
1 & 0 & 0 & 393.51 & 413.08+3.68j\tabularnewline
\hline 
1 & 1 & 0 & 468.22 & 484.95+1.83j\tabularnewline
\hline 
1 & 0 & 1 & 485.46 & 544.96+20.79j\tabularnewline
\hline 
0 & 2 & 0 & 507.46 & 522.01+2.07j\tabularnewline
\hline 
1 & 1 & 1 & 547.77 & 603.52+11.06j\tabularnewline
\hline 
0 & 0 & 2 & 568.56 & 604.31+75.40j\tabularnewline
\hline 
0 & 2 & 1 & 581.66 & 631.25+11.13j\tabularnewline
\hline 
0 & 1 & 2 & 622.60 & 666.71+46.10j\tabularnewline
\hline 
1 & 2 & 0 & 642.16 & 655+0.40j\tabularnewline
\hline 
1 & 0 & 2 & 691.46 & 743.95+28.66j\tabularnewline
\hline 
1 & 2 & 1 & 702.03 & 752.1+3.30j\tabularnewline
\hline 
\end{tabular}
\par\end{centering}

\caption{\label{tab:eigen_frequencies}The first 15 modes of closed and open
rectangular cavity, and the corresponding frequencies. $f_{\mu,\nu,\xi}=k_{\mu,\nu,\xi}c_{0}/2\pi$
for closed cavity; $F_{\mu,\nu,\xi}=K_{\mu,\nu,\xi}c_{0}/2\pi$ for
open cavity, at source frequency $f=500Hz$. }
\end{table}


\begin{table}
\begin{centering}
\begin{tabular}{ccc}
\includegraphics[scale=0.5]{\string"E:/yuhui/OneDrive/YUHUI_UWA/2D_QNM/coupled mode theory/opencavity/opencavity_theory/pictures/XY1\string".png} & \includegraphics[scale=0.5]{\string"E:/yuhui/OneDrive/YUHUI_UWA/2D_QNM/coupled mode theory/opencavity/opencavity_theory/pictures/XZ1\string".png} & \includegraphics[scale=0.5]{\string"E:/yuhui/OneDrive/YUHUI_UWA/2D_QNM/coupled mode theory/opencavity/opencavity_theory/pictures/YZ1\string".png}\tabularnewline
(0,0,0) mode $z=0$ plane & (0,0,0) mode $y=0$ plane & (0,0,0) mode $x=0$ plane\tabularnewline
\includegraphics[scale=0.5]{\string"E:/yuhui/OneDrive/YUHUI_UWA/2D_QNM/coupled mode theory/opencavity/opencavity_theory/pictures/XY2\string".png} & \includegraphics[scale=0.5]{\string"E:/yuhui/OneDrive/YUHUI_UWA/2D_QNM/coupled mode theory/opencavity/opencavity_theory/pictures/XZ2\string".png} & \includegraphics[scale=0.5]{\string"E:/yuhui/OneDrive/YUHUI_UWA/2D_QNM/coupled mode theory/opencavity/opencavity_theory/pictures/YZ2\string".png}\tabularnewline
(0,1,0) mode $z=0$ plane & (0,1,0) mode $y=0$ plane & (0,1,0) mode $x=0$ plane\tabularnewline
\includegraphics[scale=0.5]{\string"E:/yuhui/OneDrive/YUHUI_UWA/2D_QNM/coupled mode theory/opencavity/opencavity_theory/pictures/XY3\string".png} & \includegraphics[scale=0.5]{\string"E:/yuhui/OneDrive/YUHUI_UWA/2D_QNM/coupled mode theory/opencavity/opencavity_theory/pictures/XZ3\string".png} & \includegraphics[scale=0.5]{\string"E:/yuhui/OneDrive/YUHUI_UWA/2D_QNM/coupled mode theory/opencavity/opencavity_theory/pictures/YZ3\string".png}\tabularnewline
(0,0,1) mode $z=0$ plane & (0,0,1) mode $y=0$ plane & (0,0,1) mode $x=0$ plane\tabularnewline
\includegraphics[scale=0.5]{\string"E:/yuhui/OneDrive/YUHUI_UWA/2D_QNM/coupled mode theory/opencavity/opencavity_theory/pictures/XY6\string".png} & \includegraphics[scale=0.5]{\string"E:/yuhui/OneDrive/YUHUI_UWA/2D_QNM/coupled mode theory/opencavity/opencavity_theory/pictures/XZ6\string".png} & \includegraphics[scale=0.5]{\string"E:/yuhui/OneDrive/YUHUI_UWA/2D_QNM/coupled mode theory/opencavity/opencavity_theory/pictures/YZ6\string".png}\tabularnewline
(1,1,0) mode $z=0$ plane & (1,1,0) mode $y=0$ plane & (1,1,0) mode $x=0$ plane\tabularnewline
\includegraphics[scale=0.5]{\string"E:/yuhui/OneDrive/YUHUI_UWA/2D_QNM/coupled mode theory/opencavity/opencavity_theory/pictures/XY10\string".png} & \includegraphics[scale=0.5]{\string"E:/yuhui/OneDrive/YUHUI_UWA/2D_QNM/coupled mode theory/opencavity/opencavity_theory/pictures/XZ10\string".png} & \includegraphics[scale=0.5]{\string"E:/yuhui/OneDrive/YUHUI_UWA/2D_QNM/coupled mode theory/opencavity/opencavity_theory/pictures/YZ10\string".png}\tabularnewline
(0,0,2) mode $z=0$ plane & (0,0,2) mode $y=0$ plane & (0,0,2) mode $x=0$ plane\tabularnewline
\end{tabular}
\par\end{centering}

\caption{\label{tab:Modulus of modes}The modulus of $\Psi_{\mu,\nu,\xi}(\boldsymbol{x})$
for (0,0,0), (0,1,0), (0,0,1), (1,1,0), (0.0,2) modes, when source
frequency $f=500$ Hz.}
\end{table}
\begin{table}
\begin{centering}
\begin{tabular}{|c|c|c|c|c|}
\hline 
 & 1 & 2 & 3 & 4\tabularnewline
\hline 
\hline 
1 & 1 & 0 & 0 & 0\tabularnewline
\hline 
2 & 0 & 1 & 0 & 0\tabularnewline
\hline 
3 & 0 & 0 & 1 & 0\tabularnewline
\hline 
4 & 0 & 0 & 0 & 1\tabularnewline
\hline 
\end{tabular}
\par\end{centering}

\caption{\label{tab:check_biorthogonality}The bi-orthogonality of first 4
modes.}
\end{table}


\begin{figure}
\begin{centering}
\includegraphics{\string"E:/yuhui/OneDrive/YUHUI_UWA/2D_QNM/coupled mode theory/opencavity/opencavity_theory/cvg_in_out\string".png}
\par\end{centering}

\caption{\label{fig:convergence I}The calculated amplitude and phase of the
sound pressure as a function of eigenmodes orders. The sound source
is located at $(0.1,\ 0.1,\ -L_{z}+0.1)$ m with $f=500Hz$, $q_{s}=10^{-4}kg/s^{2}$;
(left) location $(0.2,\ 0.3,\ -L_{z}+0.4)$ m in the cavity and (right)
location $(1.3,\ 1.4,\ -L_{z}+1.5)$ m outside the cavity. }
\end{figure}


\begin{figure}
\begin{centering}
\includegraphics{\string"E:/yuhui/OneDrive/YUHUI_UWA/2D_QNM/coupled mode theory/opencavity/opencavity_theory/modal_coefficients\string".png}
\par\end{centering}

\caption{\label{fig:Modal_coefficients}The amplitude of modal coefficients
$|c_{\mu,\nu,\xi}|$, vs orders of mode, when source frequency is
$500Hz$.}
\end{figure}


\begin{figure}
\begin{centering}
\includegraphics{\string"E:/yuhui/OneDrive/YUHUI_UWA/2D_QNM/coupled mode theory/opencavity/opencavity_theory/prediction_20eigenmodes\string".png}
\par\end{centering}

\caption{\label{fig:Comparison_FEA_Analytical}Comparison between the sound
field obtained by the anlytical model (marked by MATLAB) and numerical
simulations (marked by COMSOL) when the excitation point source is
located at (0.1,0.1,0.1) m and $q0=10^{-4}m^{3}/s$; (a) sound pressure
level at (0.2,0.3,0.4) m in the cavity and (1.3,1.4,1.5) m outside
the cavity. }
\end{figure}



\section{Conclusions and discussion}


\section*{Acknowledgement}
\end{document}
